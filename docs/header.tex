\usepackage{a4wide}
% =============================================================================
\usepackage{ifpdf}
% usage:
% \ifpdf
%   % pdf code
% \else
%   % dvi code
% \fi

% ======== Font ===============================================================
\usepackage[T1]{fontenc}
\usepackage{slantsc}
\usepackage{amssymb}
\usepackage{amsfonts}
% \usepackage{pifont}

% ======== Theorem ============================================================
\usepackage{amsthm}
\newtheorem{theorem}{Theorem}
\newtheorem{lemma}[theorem]{Lemma}
\newtheorem{proposition}[theorem]{Proposition}
\newtheorem{observation}[theorem]{Observation}
\newtheorem{corollary}[theorem]{Corollary}
\newtheorem{definition}[theorem]{Definition}

% ======== Math ===============================================================
% Be sure to load this package with the cmex10 option to ensure that
% only type 1 fonts will utilized at all point sizes. Without this option,
% it is possible that some math symbols, particularly those within
% footnotes, will be rendered in bitmap form which will result in a
% document that can not be IEEE Xplore compliant!
\usepackage[cmex10]{amsmath}
%\usepackage{algorithmicx}
%\usepackage{algorithmic}
% Do NOT use the algorithm floating environment provided by algorithm.sty
% (by the same authors) or algorithm2e.sty (by Christophe Fiorio) as
% IEEE does not use dedicated algorithm float types and packages that
% provide these will not provide correct IEEE style captions. 

% ======== pdf, url and hyperlink =============================================
%\usepackage{url}
\usepackage{hyperref}
% Use the following instead of the previous if you use 'dvipdfmx'
% instead of dvipdf. However, 'breakurl' breaks when using 'dvipdfmx'.
% \usepackage[dvipdfmx]{hyperref}
\usepackage[hyphenbreaks]{breakurl}

% ======== Specialized list ===================================================
\usepackage{paralist} % compactenum,compactitem, inparaitem, inparaenum

% ======== Alignment ==========================================================
\usepackage{array}
%\usepackage{mdwmath}
%\usepackage{mdwtab}
%\usepackage{eqparbox}

% ======== Subfigure ==========================================================
%\usepackage[font=footnotesize]{subfig}
% subfig.sty requires and automatically loads Axel Sommerfeldt's
% caption.sty which will override IEEEtran.cls handling of captions
% and this will result in nonIEEE style figure/table captions. To
% prevent this problem, be sure and preload caption.sty with its
% "caption=false" package option. This is will preserve IEEEtran.cls
% handing of captions. Version 1.3 (2005/06/28) and later (recommended
% due to many improvements over 1.2) of subfig.sty supports the
% caption=false option directly: 
% \usepackage[caption=false,font=footnotesize]{subfig}
% \usepackage[caption=false,bf]{caption}
%% \usepackage[caption=false,font={sf,footnotesize}]{subfig}
\usepackage[caption=false,font=footnotesize]{subfig}

% ======== Float ==============================================================
%\usepackage{fixltx2e}
% fixltx2e, the successor to the earlier fix2col.sty, corrects a few
% problems in the LaTeX2e kernel, the most notable of which is that in
% current LaTeX2e releases, the ordering of single and double column
% floats is not guaranteed to be preserved. Thus, an unpatched LaTeX2e
% can allow a single column figure to be placed prior to an earlier
% double column figure. 

%\usepackage{stfloats}
% stfloats.sty gives LaTeX2e the ability to do double column floats at
% the bottom of the page as well as the top. (e.g.,
% "\begin{figure*}[!b]" is not normally possible in LaTeX2e). It also
% provides a command:
% %\fnbelowfloat
% to enable the placement of footnotes below bottom floats (the standard
% LaTeX2e kernel puts them above bottom floats). This is an invasive package
% which rewrites many portions of the LaTeX2e float routines. It may not work
% with other packages that modify the LaTeX2e float routines.
% Do not use the stfloats baselinefloat ability as IEEE
% does not allow \baselineskip to stretch. Authors submitting work to the
% IEEE should note that IEEE rarely uses double column equations and
% that authors should try to avoid such use. Do not be tempted to use the
% cuted.sty or midfloat.sty packages (also by Sigitas Tolusis) as IEEE does
% not format its papers in such ways.

%\ifCLASSOPTIONcaptionsoff
%  \usepackage[nomarkers]{endfloat}
% \let\MYoriglatexcaption\caption
% \renewcommand{\caption}[2][\relax]{\MYoriglatexcaption[#2]{#2}}
%\fi
% endfloat.sty may be useful when used in conjunction with IEEEtran.cls'
% captionsoff option. Some IEEE journals/societies require that submissions
% have lists of figures/tables at the end of the paper and that
% figures/tables without any captions are placed on a page by themselves at
% the end of the document. If needed, the draftcls IEEEtran class option or
% \CLASSINPUTbaselinestretch interface can be used to increase the line
% spacing as well. Be sure and use the nomarkers option of endfloat to
% prevent endfloat from "marking" where the figures would have been placed
% in the text. The two hack lines of code above are a slight modification of
% that suggested by in the endfloat docs (section 8.3.1) to ensure that
% the full captions always appear in the list of figures/tables - even if
% the user used the short optional argument of \caption[]{}.
% IEEE papers do not typically make use of \caption[]'s optional argument,
% so this should not be an issue. A similar trick can be used to disable
% captions of packages such as subfig.sty that lack options to turn off
% the subcaptions:
% For subfig.sty:
% \let\MYorigsubfloat\subfloat
% \renewcommand{\subfloat}[2][\relax]{\MYorigsubfloat[]{#2}}
% For subfigure.sty:
% \let\MYorigsubfigure\subfigure
% \renewcommand{\subfigure}[2][\relax]{\MYorigsubfigure[]{#2}}
% However, the above trick will not work if both optional arguments of
% the \subfloat/subfig command are used. Furthermore, there needs to be a
% description of each subfigure *somewhere* and endfloat does not add
% subfigure captions to its list of figures. Thus, the best approach is to
% avoid the use of subfigure captions (many IEEE journals avoid them anyway)
% and instead reference/explain all the subfigures within the main caption.
%
% The IEEEtran \ifCLASSOPTIONcaptionsoff conditional can also be used
% later in the document, say, to conditionally put the References on a
% page by themselves.

% \usepackage[outercaption]{sidecap}

% ======== Citations ==========================================================
\usepackage{cite}
\usepackage{nameref}

% ======== Tables =============================================================
\usepackage{multirow}
\usepackage{multicol}
\usepackage{tabularx}

% ======== Vector graphics ====================================================
\usepackage{pstricks,pst-node,pst-plot}

% ======== Listings ===========================================================
%\usepackage{alltt}
\usepackage{listings}
\lstset{%
  language=C++,%
%  basewidth=\ccBaseWidth,
  keywordstyle=\color{blue},commentstyle=\color{red}%
}
\def\myLstinline#1{\lstinline[columns=fixed]{#1}}

% ======== Miscellaneous ======================================================
\usepackage{psfrag}
\usepackage{colordvi}
\usepackage{wrapfig}
\usepackage{calc}
\usepackage{xspace}
% \usepackage{amscd}
% \usepackage{fancybox}
% \usepackage{comment}
% \usepackage{marginnote}

% ======== Math Commands ======================================================
% \renewcommand{\QED}{\hfill\QEDopen}

\def\calA{{\cal A}}
\def\calB{{\cal B}}
\def\calC{{\cal C}}
\def\calE{{\cal E}}
\def\calG{{\cal G}}
\def\calH{{\cal H}}
\def\calI{{\cal I}}
\def\calJ{{\cal J}}
\def\calK{{\cal K}}
\def\calL{{\cal L}}
\def\calM{{\cal M}}
\def\calO{{\cal O}}
\def\calP{{\cal P}}
\def\calQ{{\cal Q}}
\def\calS{{\cal S}}
\def\calR{{\cal R}}
\def\calT{{\cal T}}
\def\calU{{\cal U}}

\newcommand{\B}{\ensuremath{\mathbb{B}}}               % B boolesche Werte
\newcommand{\C}{\ensuremath{\mathbb{C}}}               % C komplexe Zahlen
\newcommand{\D}{\ensuremath{\mathbb{D}}}               % D Definitionsbereich
\newcommand{\K}{\ensuremath{\mathbb{K}}}               % K field
\newcommand{\N}{\ensuremath{\mathbb{N}}}               % N natuerliche Zahlen
\newcommand{\Q}{\ensuremath{\mathbb{Q}}}               % Q rationale Zahlen
\newcommand{\PP}{\ensuremath{\mathbb{P}}}              % P projektiver Raum
\newcommand{\R}{\ensuremath{\mathbb{R}}}               % R reelle Zahlen
\newcommand{\Z}{\ensuremath{\mathbb{Z}}}               % Z ganze Zahlen

\newcommand{\Pone}{\ensuremath{\PP \rule{0.3mm}{0mm}^1}\xspace}     % P^1

\newcommand{\Rtwo}{\ensuremath{\R \rule{0.3mm}{0mm}^2}\xspace}      % R^2
\newcommand{\Rthree}{\ensuremath{\R \rule{0.3mm}{0mm}^3}\xspace}    % R^3
\newcommand{\Rd}{\ensuremath{\R \rule{0.3mm}{0mm}^d}\xspace}        % R^d
\newcommand{\Rx}[1]{\ensuremath{\R \rule{0.3mm}{0mm} ^ {#1}}\xspace} % R^x
\newcommand{\SOd}[1]{\ensuremath{\mathbb{S}\rule{0.3mm}{0mm}^{#1}}}
\newcommand{\SOtwo}{\SOd{2}\xspace}
\newcommand{\HOd}[1]{\ensuremath{\mathbb{H}\rule{0.3mm}{0mm}^{#1}}}
\newcommand{\HOtwo}{\HOd{2}\xspace}
\def\eps{{\epsilon}}
% \newcommand{\naive}{na\"{\i}ve}
\newcommand{\Cfree}{{\mathcal C}_{\rm free}}
\newcommand{\Cforb}{{\mathcal C}_{\rm forb}}
\newcommand{\parms}{{\Phi}}      % Parameter space
\newcommand{\parmf}{{\phi}}      % Parameter function
\newcommand{\RC}{\overline{\R}}  % compactified real line
\newcommand{\Min}[1]{#1_{\rm min}}
\newcommand{\Max}[1]{#1_{\rm max}}
\newcommand{\umin}{\Min{u}}
\newcommand{\umax}{\Max{u}}
\newcommand{\vmin}{\Min{v}}
\newcommand{\vmax}{\Max{v}}
\newcommand{\vecd}{\vec{d}}

% ======== Hyphenation ========================================================
\hyphenation{op-tical net-works semi-conduc-tor}

% ======== English ============================================================
\newcommand{\ie}{i.e.,\xspace}
\newcommand{\eg}{e.g.,\xspace}
\newcommand{\etal}{et~al.\xspace}
\newcommand{\Wlog}{W.\,l.\,o.\,g.\xspace}
\newcommand{\apriori}{a~priori\xspace}
\newcommand{\iFF}{if and only if\xspace}

% ======== Extras =============================================================
\newcommand{\cgal}{{\sc Cgal}}
\newcommand{\sgal}{{\sc Sgal}}
\newcommand{\vrml}{{\sc Vrml}}
\newcommand{\boost}{{\sc Boost}}
\newcommand{\dcel}{{\sc Dcel}}
\newcommand{\gmp}{{\sc Gmp}}
\newcommand{\lego}{{\sc Lego}}
\newcommand{\zcorp}{{\sc ZCorp}}

% ======== Part colors ========================================================
\newcommand{\redpart}{{\sc \textcolor{red}{$R$}}}
\newcommand{\greenpart}{{\sc \textcolor{green}{$G$}}}
\newcommand{\bluepart}{{\sc \textcolor{blue}{$B$}}}
\newcommand{\purplepart}{{\sc \textcolor{magenta}{$P$}}}
\newcommand{\yellowpart}{{\sc \textcolor{orange}{$Y$}}}
\newcommand{\turquoisepart}{{\sc \textcolor{cyan}{$T$}}}

% ======== CGAL package names =================================================
\newcommand{\cgalPackage}[1]{{\emph{#1}\index{CGAL package@\cgal{} package!#1@\emph{#1}}}}

\newcommand{\AlgebraicFoundationsPackage}{\cgalPackage{Algebraic Foundations}}
\newcommand{\NumberTypesPackage}{\cgalPackage{Number Types}}
\newcommand{\ModularArithmeticPackage}{\cgalPackage{Modular Arithmetic}}
\newcommand{\PolynomialPackage}{\cgalPackage{Polynomial}}
\newcommand{\AlgebraicKernelPackage}{\cgalPackage{Algebraic Kernel}}
\newcommand{\MonotoneandSortedMatrixSearchPackage}{\cgalPackage{Monotone and Sorted Matrix Search}}
\newcommand{\LinearandQuadraticProgrammingSolverPackage}{\cgalPackage{Linear and Quadratic Programming Solver}}
\newcommand{\iiDandiiiDGeometryKernelPackage}{\cgalPackage{2D and 3D Geometry Kernel}}
\newcommand{\dDGeometryKernelPackage}{\cgalPackage{dD Geometry Kernel}}
\newcommand{\iiDCircularGeometryKernelPackage}{\cgalPackage{2D Circular Geometry Kernel}}
\newcommand{\iiiDSphericalGeometryKernelPackage}{\cgalPackage{3D Spherical Geometry Kernel}}
\newcommand{\iiDConvexHullsandExtremePointsPackage}{\cgalPackage{2D Convex Hulls and Extreme Points}}
\newcommand{\iiiDConvexHullsPackage}{\cgalPackage{3D Convex Hulls}}
\newcommand{\dDConvexHullsandDelaunayTriangulationsPackage}{\cgalPackage{dD Convex Hulls and Delaunay Triangulations}}
\newcommand{\iiDPolygonsPackage}{\cgalPackage{2D Polygons}}
\newcommand{\iiDRegularizedBooleanSetOperationsPackage}{\cgalPackage{2D Regularized Boolean Set-Operations}}
\newcommand{\iiDBooleanOperationsonNefPolygonsPackage}{\cgalPackage{2D Boolean Operations on Nef Polygons}}
\newcommand{\iiDBooleanOperationsonNefPolygonsEmbeddedontheSpherePackage}{\cgalPackage{2D Boolean Operations on Nef Polygons Embedded on the Sphere}}
\newcommand{\iiDPolygonPartitioningPackage}{\cgalPackage{2D Polygon Partitioning}}
\newcommand{\iiDStraightSkeletonandPolygonOffsettingPackage}{\cgalPackage{2D Straight Skeleton and Polygon Offsetting}}
\newcommand{\iiDMinkowskiSumsPackage}{\cgalPackage{2D Minkowski Sums}}
\newcommand{\iiiDPolyhedralSurfacesPackage}{\cgalPackage{3D Polyhedral Surfaces}}
\newcommand{\HalfedgeDataStructuresPackage}{\cgalPackage{Halfedge Data Structures}}
\newcommand{\iiiDBooleanOperationsonNefPolyhedraPackage}{\cgalPackage{3D Boolean Operations on Nef Polyhedra}}
\newcommand{\ConvexDecompositionofPolyhedraPackage}{\cgalPackage{Convex Decomposition of Polyhedra}}
\newcommand{\iiiDMinkowskiSumofPolyhedraPackage}{\cgalPackage{3D Minkowski Sum of Polyhedra}}
\newcommand{\iiDArrangementsPackage}{\cgalPackage{2D Arrangements}}
\newcommand{\iiDIntersectionofCurvesPackage}{\cgalPackage{2D Intersection of Curves}}
\newcommand{\iiDSnapRoundingPackage}{\cgalPackage{2D Snap Rounding}}
\newcommand{\EnvelopesofCurvesiniiDPackage}{\cgalPackage{Envelopes of Curves in 2D}}
\newcommand{\EnvelopesofSurfacesiniiiDPackage}{\cgalPackage{Envelopes of Surfaces in 3D}}
\newcommand{\iiDTriangulationsPackage}{\cgalPackage{2D Triangulations}}
\newcommand{\iiDTriangulationDataStructurePackage}{\cgalPackage{2D Triangulation Data Structure}}
\newcommand{\iiiDTriangulationsPackage}{\cgalPackage{3D Triangulations}}
\newcommand{\iiiDTriangulationDataStructurePackage}{\cgalPackage{3D Triangulation Data Structure}}
\newcommand{\iiiDPeriodicTriangulationsPackage}{\cgalPackage{3D Periodic Triangulations}}
\newcommand{\iiDAlphaShapesPackage}{\cgalPackage{2D Alpha Shapes}}
\newcommand{\iiiDAlphaShapesPackage}{\cgalPackage{3D Alpha Shapes}}
\newcommand{\iiDSegmentDelaunayGraphsPackage}{\cgalPackage{2D Segment Delaunay Graphs}}
\newcommand{\iiDApolloniusGraphsPackage}{\cgalPackage{2D Apollonius Graphs (Delaunay Graphs of Disks)}}
\newcommand{\iiDVoronoiDiagramAdaptorPackage}{\cgalPackage{2D Voronoi Diagram Adaptor}}
\newcommand{\iiDConformingTriangulationsandMeshesPackage}{\cgalPackage{2D Conforming Triangulations and Meshes}}
\newcommand{\iiiDSurfaceMeshGenerationPackage}{\cgalPackage{3D Surface Mesh Generation}}
\newcommand{\SurfaceReconstructionfromPointSetsPackage}{\cgalPackage{Surface Reconstruction from Point Sets}}
\newcommand{\iiiDSkinSurfaceMeshingPackage}{\cgalPackage{3D Skin Surface Meshing}}
\newcommand{\iiiDMeshGenerationPackage}{\cgalPackage{3D Mesh Generation}}
\newcommand{\iiiDSurfaceSubdivisionMethodsPackage}{\cgalPackage{3D Surface Subdivision Methods}}
\newcommand{\TriangulatedSurfaceMeshSimplificationPackage}{\cgalPackage{Triangulated Surface Mesh Simplification}}
\newcommand{\PlanarParameterizationofTriangulatedSurfaceMeshesPackage}{\cgalPackage{Planar Parameterization of Triangulated Surface Meshes}}
\newcommand{\iiDPlacementofStreamlinesPackage}{\cgalPackage{2D Placement of Streamlines}}
\newcommand{\ApproximationofRidgesandUmbilicsonTriangulatedSurfaceMeshesPackage}{\cgalPackage{Approximation of Ridges and Umbilics on Triangulated Surface Meshes }}
\newcommand{\EstimationofLocalDifferentialPropertiesPackage}{\cgalPackage{Estimation of Local Differential Properties}}
\newcommand{\PointSetProcessingPackage}{\cgalPackage{Point Set Processing}}
\newcommand{\iiDRangeandNeighborSearchPackage}{\cgalPackage{2D Range and Neighbor Search}}
\newcommand{\IntervalSkipListPackage}{\cgalPackage{Interval Skip List}}
\newcommand{\dDSpatialSearchingPackage}{\cgalPackage{dD Spatial Searching}}
\newcommand{\dDRangeandSegmentTreesPackage}{\cgalPackage{dD Range and Segment Trees}}
\newcommand{\IntersectingSequencesofdDIsoorientedBoxesPackage}{\cgalPackage{Intersecting Sequences of dD Iso-oriented Boxes}}
\newcommand{\AABBTreePackage}{\cgalPackage{AABB Tree}}
\newcommand{\SpatialSortingPackage}{\cgalPackage{Spatial Sorting}}
\newcommand{\BoundingVolumesPackage}{\cgalPackage{Bounding Volumes}}
\newcommand{\InscribedAreasPackage}{\cgalPackage{Inscribed Areas}}
\newcommand{\OptimalDistancesPackage}{\cgalPackage{Optimal Distances}}
\newcommand{\PrincipalComponentAnalysisPackage}{\cgalPackage{Principal Component Analysis}}
\newcommand{\iiDandSurfaceFunctionInterpolationPackage}{\cgalPackage{2D and Surface Function Interpolation}}
\newcommand{\KineticDataStructuresPackage}{\cgalPackage{Kinetic Data Structures}}
\newcommand{\KineticFrameworkPackage}{\cgalPackage{Kinetic Framework}}
\newcommand{\STLExtensionsforCGALPackage}{\cgalPackage{STL Extensions for CGAL}}
\newcommand{\CGALandtheBoostGraphLibraryPackage}{\cgalPackage{CGAL and the Boost Graph Library}}
\newcommand{\CGALandBoostPropertyMapsPackage}{\cgalPackage{CGAL and Boost Property Maps}}
\newcommand{\HandlesandCirculatorsPackage}{\cgalPackage{Handles and Circulators}}
\newcommand{\GeometricObjectGeneratorsPackage}{\cgalPackage{Geometric Object Generators}}
\newcommand{\ProfilingtoolsHashMapUnionfindModifiersPackage}{\cgalPackage{Profiling tools, Hash Map, Union-find, Modifiers}}
\newcommand{\IOStreamsPackage}{\cgalPackage{IO Streams}}
\newcommand{\GeomviewPackage}{\cgalPackage{Geomview}}
\newcommand{\CGALandtheQtGraphicsViewFrameworkPackage}{\cgalPackage{CGAL and the Qt Graphics View Framework}}
\newcommand{\CGALIpeletsPackage}{\cgalPackage{CGAL Ipelets}}

\newcommand{\iiiDLinesThroughSegments}{\cgalPackage{3D Lines Through Segments}}

% ======== Graphics ===========================================================
\usepackage{graphicx}
% \usepackage{psfig}
% \usepackage{epsfig}
\DeclareGraphicsExtensions{.png}
\DeclareGraphicsExtensions{.jpg}
\DeclareGraphicsExtensions{.svg}
% \DeclareGraphicsRule{.png}{eps}{.bb}{`convert -compress JPEG #1 eps2:-}
\DeclareGraphicsRule{.png}{eps}{.bb}{`convert #1 eps2:-}
\DeclareGraphicsRule{.jpg}{eps}{.bb}{`convert #1 eps2:-}
\DeclareGraphicsRule{.svg}{eps}{.bb}{`convert #1 eps2:-}

% ======== Editing ============================================================
\usepackage{ifthen}
\newboolean{ShowTODO}
\setboolean{ShowTODO}{true}

%% \newcommand{\cpm}{\textsc{Cpm}}
%% \newcounter{copyrightbox}
%% \renewcommand{\captionfont}{\sffamily}
