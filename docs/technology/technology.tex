\documentclass[11pt,a4paper]{article}
\usepackage{a4wide}
% =============================================================================
\usepackage{ifpdf}
% usage:
% \ifpdf
%   % pdf code
% \else
%   % dvi code
% \fi

% ======== Font ===============================================================
\usepackage[T1]{fontenc}
\usepackage{lmodern}
\usepackage{slantsc}
\usepackage{amssymb}
\usepackage{amsfonts}
% \usepackage{pifont}

% ======== Theorem ============================================================
\usepackage{amsthm}
\newtheorem{theorem}{Theorem}
\newtheorem{lemma}[theorem]{Lemma}
\newtheorem{proposition}[theorem]{Proposition}
\newtheorem{observation}[theorem]{Observation}
\newtheorem{corollary}[theorem]{Corollary}
\newtheorem{definition}[theorem]{Definition}

% ======== Math ===============================================================
% Be sure to load this package with the cmex10 option to ensure that
% only type 1 fonts will utilized at all point sizes. Without this option,
% it is possible that some math symbols, particularly those within
% footnotes, will be rendered in bitmap form which will result in a
% document that can not be IEEE Xplore compliant!
\usepackage[cmex10]{amsmath}
%\usepackage{algorithmicx}
%\usepackage{algorithmic}
% Do NOT use the algorithm floating environment provided by algorithm.sty
% (by the same authors) or algorithm2e.sty (by Christophe Fiorio) as
% IEEE does not use dedicated algorithm float types and packages that
% provide these will not provide correct IEEE style captions. 

% ======== pdf, url and hyperlink =============================================
%\usepackage{url}
\usepackage{hyperref}
% Use the following instead of the previous if you use 'dvipdfmx'
% instead of dvipdf. However, 'breakurl' breaks when using 'dvipdfmx'.
% \usepackage[dvipdfmx]{hyperref}
\usepackage[hyphenbreaks]{breakurl}

% ======== Specialized list ===================================================
\usepackage{paralist} % compactenum,compactitem, inparaitem, inparaenum

% ======== Alignment ==========================================================
\usepackage{array}
%\usepackage{mdwmath}
%\usepackage{mdwtab}
%\usepackage{eqparbox}

% ======== Subfigure ==========================================================
%\usepackage[font=footnotesize]{subfig}
% subfig.sty requires and automatically loads Axel Sommerfeldt's
% caption.sty which will override IEEEtran.cls handling of captions
% and this will result in nonIEEE style figure/table captions. To
% prevent this problem, be sure and preload caption.sty with its
% "caption=false" package option. This is will preserve IEEEtran.cls
% handing of captions. Version 1.3 (2005/06/28) and later (recommended
% due to many improvements over 1.2) of subfig.sty supports the
% caption=false option directly: 
% \usepackage[caption=false,font=footnotesize]{subfig}
% \usepackage[caption=false,bf]{caption}
%% \usepackage[caption=false,font={sf,footnotesize}]{subfig}
\usepackage[caption=false,font=footnotesize]{subfig}

% ======== Float ==============================================================
%\usepackage{fixltx2e}
% fixltx2e, the successor to the earlier fix2col.sty, corrects a few
% problems in the LaTeX2e kernel, the most notable of which is that in
% current LaTeX2e releases, the ordering of single and double column
% floats is not guaranteed to be preserved. Thus, an unpatched LaTeX2e
% can allow a single column figure to be placed prior to an earlier
% double column figure. 

%\usepackage{stfloats}
% stfloats.sty gives LaTeX2e the ability to do double column floats at
% the bottom of the page as well as the top. (e.g.,
% "\begin{figure*}[!b]" is not normally possible in LaTeX2e). It also
% provides a command:
% %\fnbelowfloat
% to enable the placement of footnotes below bottom floats (the standard
% LaTeX2e kernel puts them above bottom floats). This is an invasive package
% which rewrites many portions of the LaTeX2e float routines. It may not work
% with other packages that modify the LaTeX2e float routines.
% Do not use the stfloats baselinefloat ability as IEEE
% does not allow \baselineskip to stretch. Authors submitting work to the
% IEEE should note that IEEE rarely uses double column equations and
% that authors should try to avoid such use. Do not be tempted to use the
% cuted.sty or midfloat.sty packages (also by Sigitas Tolusis) as IEEE does
% not format its papers in such ways.

%\ifCLASSOPTIONcaptionsoff
%  \usepackage[nomarkers]{endfloat}
% \let\MYoriglatexcaption\caption
% \renewcommand{\caption}[2][\relax]{\MYoriglatexcaption[#2]{#2}}
%\fi
% endfloat.sty may be useful when used in conjunction with IEEEtran.cls'
% captionsoff option. Some IEEE journals/societies require that submissions
% have lists of figures/tables at the end of the paper and that
% figures/tables without any captions are placed on a page by themselves at
% the end of the document. If needed, the draftcls IEEEtran class option or
% \CLASSINPUTbaselinestretch interface can be used to increase the line
% spacing as well. Be sure and use the nomarkers option of endfloat to
% prevent endfloat from "marking" where the figures would have been placed
% in the text. The two hack lines of code above are a slight modification of
% that suggested by in the endfloat docs (section 8.3.1) to ensure that
% the full captions always appear in the list of figures/tables - even if
% the user used the short optional argument of \caption[]{}.
% IEEE papers do not typically make use of \caption[]'s optional argument,
% so this should not be an issue. A similar trick can be used to disable
% captions of packages such as subfig.sty that lack options to turn off
% the subcaptions:
% For subfig.sty:
% \let\MYorigsubfloat\subfloat
% \renewcommand{\subfloat}[2][\relax]{\MYorigsubfloat[]{#2}}
% For subfigure.sty:
% \let\MYorigsubfigure\subfigure
% \renewcommand{\subfigure}[2][\relax]{\MYorigsubfigure[]{#2}}
% However, the above trick will not work if both optional arguments of
% the \subfloat/subfig command are used. Furthermore, there needs to be a
% description of each subfigure *somewhere* and endfloat does not add
% subfigure captions to its list of figures. Thus, the best approach is to
% avoid the use of subfigure captions (many IEEE journals avoid them anyway)
% and instead reference/explain all the subfigures within the main caption.
%
% The IEEEtran \ifCLASSOPTIONcaptionsoff conditional can also be used
% later in the document, say, to conditionally put the References on a
% page by themselves.

% \usepackage[outercaption]{sidecap}

% ======== Citations ==========================================================
\usepackage{cite}
\usepackage{nameref}

% ======== Tables =============================================================
\usepackage{multirow}
\usepackage{multicol}
\usepackage{tabularx}

% ======== Vector graphics ====================================================
\usepackage{pstricks,pst-node,pst-plot}

% ======== Listings ===========================================================
%\usepackage{alltt}
\usepackage{listings}
\lstset{%
  language=C++,%
%  basewidth=\ccBaseWidth,
  keywordstyle=\color{blue},commentstyle=\color{red}%
}
\def\myLstinline#1{\lstinline[columns=fixed]{#1}}

% ======== Miscellaneous ======================================================
\usepackage{psfrag}
\usepackage{colordvi}
\usepackage{wrapfig}
\usepackage{calc}
\usepackage{xspace}
% \usepackage{amscd}
% \usepackage{fancybox}
% \usepackage{comment}
% \usepackage{marginnote}

% ======== Math Commands ======================================================
% \renewcommand{\QED}{\hfill\QEDopen}

\def\calA{{\cal A}}
\def\calB{{\cal B}}
\def\calC{{\cal C}}
\def\calE{{\cal E}}
\def\calG{{\cal G}}
\def\calH{{\cal H}}
\def\calI{{\cal I}}
\def\calJ{{\cal J}}
\def\calK{{\cal K}}
\def\calL{{\cal L}}
\def\calM{{\cal M}}
\def\calO{{\cal O}}
\def\calP{{\cal P}}
\def\calQ{{\cal Q}}
\def\calS{{\cal S}}
\def\calR{{\cal R}}
\def\calT{{\cal T}}
\def\calU{{\cal U}}

\newcommand{\B}{\ensuremath{\mathbb{B}}}               % B boolesche Werte
\newcommand{\C}{\ensuremath{\mathbb{C}}}               % C komplexe Zahlen
\newcommand{\D}{\ensuremath{\mathbb{D}}}               % D Definitionsbereich
\newcommand{\K}{\ensuremath{\mathbb{K}}}               % K field
\newcommand{\N}{\ensuremath{\mathbb{N}}}               % N natuerliche Zahlen
\newcommand{\Q}{\ensuremath{\mathbb{Q}}}               % Q rationale Zahlen
\newcommand{\PP}{\ensuremath{\mathbb{P}}}              % P projektiver Raum
\newcommand{\R}{\ensuremath{\mathbb{R}}}               % R reelle Zahlen
\newcommand{\Z}{\ensuremath{\mathbb{Z}}}               % Z ganze Zahlen

\newcommand{\Pone}{\ensuremath{\PP \rule{0.3mm}{0mm}^1}\xspace}     % P^1

\newcommand{\Rtwo}{\ensuremath{\R \rule{0.3mm}{0mm}^2}\xspace}      % R^2
\newcommand{\Rthree}{\ensuremath{\R \rule{0.3mm}{0mm}^3}\xspace}    % R^3
\newcommand{\Rd}{\ensuremath{\R \rule{0.3mm}{0mm}^d}\xspace}        % R^d
\newcommand{\Rx}[1]{\ensuremath{\R \rule{0.3mm}{0mm} ^ {#1}}\xspace} % R^x
\newcommand{\SOd}[1]{\ensuremath{\mathbb{S}\rule{0.3mm}{0mm}^{#1}}}
\newcommand{\SOtwo}{\SOd{2}\xspace}
\newcommand{\HOd}[1]{\ensuremath{\mathbb{H}\rule{0.3mm}{0mm}^{#1}}}
\newcommand{\HOtwo}{\HOd{2}\xspace}
\def\eps{{\epsilon}}
% \newcommand{\naive}{na\"{\i}ve}
\newcommand{\Cfree}{{\mathcal C}_{\rm free}}
\newcommand{\Cforb}{{\mathcal C}_{\rm forb}}
\newcommand{\parms}{{\Phi}}      % Parameter space
\newcommand{\parmf}{{\phi}}      % Parameter function
\newcommand{\RC}{\overline{\R}}  % compactified real line
\newcommand{\Min}[1]{#1_{\rm min}}
\newcommand{\Max}[1]{#1_{\rm max}}
\newcommand{\umin}{\Min{u}}
\newcommand{\umax}{\Max{u}}
\newcommand{\vmin}{\Min{v}}
\newcommand{\vmax}{\Max{v}}
\newcommand{\vecd}{\vec{d}}

% ======== Hyphenation ========================================================
\hyphenation{op-tical net-works semi-conduc-tor}

% ======== English ============================================================
\newcommand{\ie}{i.e.,\xspace}
\newcommand{\eg}{e.g.,\xspace}
\newcommand{\etal}{et~al.\xspace}
\newcommand{\Wlog}{W.\,l.\,o.\,g.\xspace}
\newcommand{\apriori}{a~priori\xspace}
\newcommand{\iFF}{if and only if\xspace}

% ======== Extras =============================================================
\newcommand{\cgal}{{\sc Cgal}}
\newcommand{\sgal}{{\sc Sgal}}
\newcommand{\vrml}{{\sc Vrml}}
\newcommand{\boost}{{\sc Boost}}
\newcommand{\dcel}{{\sc Dcel}}
\newcommand{\gmp}{{\sc Gmp}}
\newcommand{\lego}{{\sc Lego}}
\newcommand{\ego}{{\sc Ego}}
\newcommand{\zcorp}{{\sc ZCorp}}
\newcommand{\lux}{{\sc Lux}}
\newcommand{\gartner}{{\sc Gartner}}
\newcommand{\technavio}{{\sc TechNavio}}

% ======== Part colors ========================================================
\newcommand{\redpart}{{\sc \textcolor{red}{$R$}}}
\newcommand{\greenpart}{{\sc \textcolor{green}{$G$}}}
\newcommand{\bluepart}{{\sc \textcolor{blue}{$B$}}}
\newcommand{\purplepart}{{\sc \textcolor{magenta}{$P$}}}
\newcommand{\yellowpart}{{\sc \textcolor{orange}{$Y$}}}
\newcommand{\turquoisepart}{{\sc \textcolor{cyan}{$T$}}}

% ======== CGAL package names =================================================
\newcommand{\cgalPackage}[1]{{\emph{#1}\index{CGAL package@\cgal{} package!#1@\emph{#1}}}}

\newcommand{\AlgebraicFoundationsPackage}{\cgalPackage{Algebraic Foundations}}
\newcommand{\NumberTypesPackage}{\cgalPackage{Number Types}}
\newcommand{\ModularArithmeticPackage}{\cgalPackage{Modular Arithmetic}}
\newcommand{\PolynomialPackage}{\cgalPackage{Polynomial}}
\newcommand{\AlgebraicKernelPackage}{\cgalPackage{Algebraic Kernel}}
\newcommand{\MonotoneandSortedMatrixSearchPackage}{\cgalPackage{Monotone and Sorted Matrix Search}}
\newcommand{\LinearandQuadraticProgrammingSolverPackage}{\cgalPackage{Linear and Quadratic Programming Solver}}
\newcommand{\iiDandiiiDGeometryKernelPackage}{\cgalPackage{2D and 3D Geometry Kernel}}
\newcommand{\dDGeometryKernelPackage}{\cgalPackage{dD Geometry Kernel}}
\newcommand{\iiDCircularGeometryKernelPackage}{\cgalPackage{2D Circular Geometry Kernel}}
\newcommand{\iiiDSphericalGeometryKernelPackage}{\cgalPackage{3D Spherical Geometry Kernel}}
\newcommand{\iiDConvexHullsandExtremePointsPackage}{\cgalPackage{2D Convex Hulls and Extreme Points}}
\newcommand{\iiiDConvexHullsPackage}{\cgalPackage{3D Convex Hulls}}
\newcommand{\dDConvexHullsandDelaunayTriangulationsPackage}{\cgalPackage{dD Convex Hulls and Delaunay Triangulations}}
\newcommand{\iiDPolygonsPackage}{\cgalPackage{2D Polygons}}
\newcommand{\iiDRegularizedBooleanSetOperationsPackage}{\cgalPackage{2D Regularized Boolean Set-Operations}}
\newcommand{\iiDBooleanOperationsonNefPolygonsPackage}{\cgalPackage{2D Boolean Operations on Nef Polygons}}
\newcommand{\iiDBooleanOperationsonNefPolygonsEmbeddedontheSpherePackage}{\cgalPackage{2D Boolean Operations on Nef Polygons Embedded on the Sphere}}
\newcommand{\iiDPolygonPartitioningPackage}{\cgalPackage{2D Polygon Partitioning}}
\newcommand{\iiDStraightSkeletonandPolygonOffsettingPackage}{\cgalPackage{2D Straight Skeleton and Polygon Offsetting}}
\newcommand{\iiDMinkowskiSumsPackage}{\cgalPackage{2D Minkowski Sums}}
\newcommand{\iiiDPolyhedralSurfacesPackage}{\cgalPackage{3D Polyhedral Surfaces}}
\newcommand{\HalfedgeDataStructuresPackage}{\cgalPackage{Halfedge Data Structures}}
\newcommand{\iiiDBooleanOperationsonNefPolyhedraPackage}{\cgalPackage{3D Boolean Operations on Nef Polyhedra}}
\newcommand{\ConvexDecompositionofPolyhedraPackage}{\cgalPackage{Convex Decomposition of Polyhedra}}
\newcommand{\iiiDMinkowskiSumofPolyhedraPackage}{\cgalPackage{3D Minkowski Sum of Polyhedra}}
\newcommand{\iiDArrangementsPackage}{\cgalPackage{2D Arrangements}}
\newcommand{\iiDIntersectionofCurvesPackage}{\cgalPackage{2D Intersection of Curves}}
\newcommand{\iiDSnapRoundingPackage}{\cgalPackage{2D Snap Rounding}}
\newcommand{\EnvelopesofCurvesiniiDPackage}{\cgalPackage{Envelopes of Curves in 2D}}
\newcommand{\EnvelopesofSurfacesiniiiDPackage}{\cgalPackage{Envelopes of Surfaces in 3D}}
\newcommand{\iiDTriangulationsPackage}{\cgalPackage{2D Triangulations}}
\newcommand{\iiDTriangulationDataStructurePackage}{\cgalPackage{2D Triangulation Data Structure}}
\newcommand{\iiiDTriangulationsPackage}{\cgalPackage{3D Triangulations}}
\newcommand{\iiiDTriangulationDataStructurePackage}{\cgalPackage{3D Triangulation Data Structure}}
\newcommand{\iiiDPeriodicTriangulationsPackage}{\cgalPackage{3D Periodic Triangulations}}
\newcommand{\iiDAlphaShapesPackage}{\cgalPackage{2D Alpha Shapes}}
\newcommand{\iiiDAlphaShapesPackage}{\cgalPackage{3D Alpha Shapes}}
\newcommand{\iiDSegmentDelaunayGraphsPackage}{\cgalPackage{2D Segment Delaunay Graphs}}
\newcommand{\iiDApolloniusGraphsPackage}{\cgalPackage{2D Apollonius Graphs (Delaunay Graphs of Disks)}}
\newcommand{\iiDVoronoiDiagramAdaptorPackage}{\cgalPackage{2D Voronoi Diagram Adaptor}}
\newcommand{\iiDConformingTriangulationsandMeshesPackage}{\cgalPackage{2D Conforming Triangulations and Meshes}}
\newcommand{\iiiDSurfaceMeshGenerationPackage}{\cgalPackage{3D Surface Mesh Generation}}
\newcommand{\SurfaceReconstructionfromPointSetsPackage}{\cgalPackage{Surface Reconstruction from Point Sets}}
\newcommand{\iiiDSkinSurfaceMeshingPackage}{\cgalPackage{3D Skin Surface Meshing}}
\newcommand{\iiiDMeshGenerationPackage}{\cgalPackage{3D Mesh Generation}}
\newcommand{\iiiDSurfaceSubdivisionMethodsPackage}{\cgalPackage{3D Surface Subdivision Methods}}
\newcommand{\TriangulatedSurfaceMeshSimplificationPackage}{\cgalPackage{Triangulated Surface Mesh Simplification}}
\newcommand{\PlanarParameterizationofTriangulatedSurfaceMeshesPackage}{\cgalPackage{Planar Parameterization of Triangulated Surface Meshes}}
\newcommand{\iiDPlacementofStreamlinesPackage}{\cgalPackage{2D Placement of Streamlines}}
\newcommand{\ApproximationofRidgesandUmbilicsonTriangulatedSurfaceMeshesPackage}{\cgalPackage{Approximation of Ridges and Umbilics on Triangulated Surface Meshes }}
\newcommand{\EstimationofLocalDifferentialPropertiesPackage}{\cgalPackage{Estimation of Local Differential Properties}}
\newcommand{\PointSetProcessingPackage}{\cgalPackage{Point Set Processing}}
\newcommand{\iiDRangeandNeighborSearchPackage}{\cgalPackage{2D Range and Neighbor Search}}
\newcommand{\IntervalSkipListPackage}{\cgalPackage{Interval Skip List}}
\newcommand{\dDSpatialSearchingPackage}{\cgalPackage{dD Spatial Searching}}
\newcommand{\dDRangeandSegmentTreesPackage}{\cgalPackage{dD Range and Segment Trees}}
\newcommand{\IntersectingSequencesofdDIsoorientedBoxesPackage}{\cgalPackage{Intersecting Sequences of dD Iso-oriented Boxes}}
\newcommand{\AABBTreePackage}{\cgalPackage{AABB Tree}}
\newcommand{\SpatialSortingPackage}{\cgalPackage{Spatial Sorting}}
\newcommand{\BoundingVolumesPackage}{\cgalPackage{Bounding Volumes}}
\newcommand{\InscribedAreasPackage}{\cgalPackage{Inscribed Areas}}
\newcommand{\OptimalDistancesPackage}{\cgalPackage{Optimal Distances}}
\newcommand{\PrincipalComponentAnalysisPackage}{\cgalPackage{Principal Component Analysis}}
\newcommand{\iiDandSurfaceFunctionInterpolationPackage}{\cgalPackage{2D and Surface Function Interpolation}}
\newcommand{\KineticDataStructuresPackage}{\cgalPackage{Kinetic Data Structures}}
\newcommand{\KineticFrameworkPackage}{\cgalPackage{Kinetic Framework}}
\newcommand{\STLExtensionsforCGALPackage}{\cgalPackage{STL Extensions for CGAL}}
\newcommand{\CGALandtheBoostGraphLibraryPackage}{\cgalPackage{CGAL and the Boost Graph Library}}
\newcommand{\CGALandBoostPropertyMapsPackage}{\cgalPackage{CGAL and Boost Property Maps}}
\newcommand{\HandlesandCirculatorsPackage}{\cgalPackage{Handles and Circulators}}
\newcommand{\GeometricObjectGeneratorsPackage}{\cgalPackage{Geometric Object Generators}}
\newcommand{\ProfilingtoolsHashMapUnionfindModifiersPackage}{\cgalPackage{Profiling tools, Hash Map, Union-find, Modifiers}}
\newcommand{\IOStreamsPackage}{\cgalPackage{IO Streams}}
\newcommand{\GeomviewPackage}{\cgalPackage{Geomview}}
\newcommand{\CGALandtheQtGraphicsViewFrameworkPackage}{\cgalPackage{CGAL and the Qt Graphics View Framework}}
\newcommand{\CGALIpeletsPackage}{\cgalPackage{CGAL Ipelets}}

\newcommand{\iiiDLinesThroughSegments}{\cgalPackage{3D Lines Through Segments}}

% ======== Graphics ===========================================================
\usepackage{graphicx}
% \usepackage{psfig}
% \usepackage{epsfig}
\DeclareGraphicsExtensions{.png}
\DeclareGraphicsExtensions{.jpg}
\DeclareGraphicsExtensions{.svg}
% \DeclareGraphicsRule{.png}{eps}{.bb}{`convert -compress JPEG #1 eps2:-}
\DeclareGraphicsRule{.png}{eps}{.bb}{`convert #1 eps2:-}
\DeclareGraphicsRule{.jpg}{eps}{.bb}{`convert #1 eps2:-}
\DeclareGraphicsRule{.svg}{eps}{.bb}{`convert #1 eps2:-}

% ======== Editing ============================================================
\usepackage{ifthen}
\newboolean{ShowTODO}
\setboolean{ShowTODO}{true}

%% \newcommand{\cpm}{\textsc{Cpm}}
%% \newcounter{copyrightbox}
%% \renewcommand{\captionfont}{\sffamily}

\catcode`\_=11\relax
\newcommand\email[1]{\_email #1\q_nil}
\def\_email#1@#2\q_nil{%
  \href{mailto:#1@#2}{{\emailfont #1\emailampersat #2}}
}
\newcommand\emailfont{\sffamily}
\newcommand\emailampersat{{\color{red}\small@}}
\catcode`\_=8\relax   

\usepackage[margin=1in]{geometry}
% ======== Author comments ====================================================
\ifthenelse{\boolean{ShowTODO}}{%
  \def\marrow{{\raggedright\footnotesize $\longleftarrow$}}
  \def\danny#1{\textcolor{red}{{\sc Danny says: }{\marrow\sf #1}}}
  \def\efi#1{\textcolor{red}{{\sc Efi says: }{\marrow\sf #1}}}
}{%
  \def\danny#1{}
  \def\efi#1{}
}
% ======= figs ================================================================
\graphicspath{{../../figs/}}
%
% =============================================================================
\usepackage{pst-node}
\usepackage{pst-lens}
\newpsstyle{SimpleHandle}{fillstyle=solid,fillcolor=white,framearc=0.5}
% =============================================================================
\usepackage{bookmark}
\usepackage{pdflscape}
\hypersetup{
    bookmarks=true,         % show bookmarks bar?
    unicode=false,          % non-Latin characters in Acrobat�s bookmarks
    pdftoolbar=true,        % show Acrobat's toolbar?
    pdfmenubar=true,        % show Acrobat's menu?
    pdffitwindow=false,     % window fit to page when opened
    pdfstartview={FitH},    % fits the width of the page to the window
    pdftitle={Technological Strength}, % title
    pdfauthor={Efi Fogel and Dan Halperin},  % author
    pdfsubject={Technological Strength}, % subject of the document
    pdfcreator={Efi Fogel}, % creator of the document
    pdfproducer={Efi Fogel}, % producer of the document
    pdfkeywords={3D printing}, % list of keywords
    pdfnewwindow=true,      % links in new window
    colorlinks=true,        % false: boxed links; true: colored links
    linkcolor=blue,         % color of internal links
    citecolor=green,        % color of links to bibliography
    filecolor=magenta,      % color of file links
    urlcolor=cyan           % color of external links
}

\pdfpagewidth 8.5in
\pdfpageheight 11in
\usepackage{fancyhdr}
\pagestyle{empty}
% =============================================================================
\usepackage{draftwatermark}
\SetWatermarkText{Confidential}
\SetWatermarkScale{1}
% =============================================================================
\begin{document}
\title{Technological Strength}
\author{Efi Fogel\thanks{+972 52 6225863, \email{efifogel@gmail.com}}~~and Dan Halperin\thanks{+972 54 7406478, \email{halperin.dan@gmail.com}}}
\date{\today\\~\\\copyright{} All copyrights reserved to the authors}
\maketitle
% =============================================================================
\section{Introduction}
\label{sec:introduction}
% =============================================================================
\paragraph{Polygon Mesh}
A polygon mesh, also referred to as an unstructured grid, is a
collection of vertices, edges, and facets, defining the shape of a
polyhedral object in space. The facets usually consist of triangles,
quadrilaterals, or other convex polygons, since this simplifies the
application of operations on the mesh, but may also be composed of
concave polygons, polygons with holes, and degenerate polygons.
A polygon mesh consist of the geometric information and optionaly
sufficient topologic information, so that the incident relation
between the vertices, edges, and facets can be retrieved.

\paragraph{3D Model}
A 3D model is a mathematical representation of any three-dimensional
entitiy (either inanimate or living) via specialized software.
Almost all 3D models can be divided into two categories as follows:
\begin{description}
\item[Solid models] describe the volume of the objects they represent.
\item[Boundary models] describe the boundary surface of the objects
  they represent. A polygonal mesh is an example of a boundary model.
\end{description}
Solid models are more realistic, but at the same time more difficult
to construct. They are mostly used for medical and engineering
simulations, for CAD and specialized visual applications such as ray
tracing and constructive solid geometry. Boundary models are easier to
work with than solid models. Almost all visual models used in games and
film are boundary models, and all models fed to conventional 3D printers
are polygonal meshes, for example, in the STL (Stereolithography)
file-format.\footnote{See, for example,
\url{http://en.wikipedia.org/wiki/STL_(file_format)}}.

\paragraph{3D Printable Model}
Correct and consistent models are required by most conventional
applications that apply operations such as modeling, simulation,
visualization, and finite element analysis to name a few. Conventional
3D printers can produce correct 3D physical objects only when fed with
correct and consistent models. However, in the case of 3D printing, an
input model must satisfy additional requirements. First, an input
model must consist of a closed and orientable
2D-manifold,\footnote{Non-orientable 2D-manifolds, such as a Klein
  bottle, and non-closed 2D-manifolds, such as a M\"obius strip
  comprise illegal input, as they do not define proper solids.} or in
simple words, the model must be watertight (which simply implies that
if you were to pour water into your 3D model, it would hold the water
you poured into it.) In some cases, it is possible to relax the
aformentioned requirement, and alow the model to consist of a set of
closed and orientable 2D-manifolds that bind pair-wise interior disjoint
volumes. (In simple words, such a model can be decomposed into several
watertight solids that do not intersect in their interior pair-wise.)
However, none of the components may float in midair defeating gravity.
Secondly, an input model must satisfy additional requirements imposed by
the designated 3D printing device. For example, there exists a convex
decomposition, such that the Z-directional (vertical) and the
XY-directional (horizontal) widths of every convex piece are larger than
two given threshold, respectively. In simple words, all parts of the
model must be sufficiently thick with respect to the resolution of the
printer. Finally, the model must be bounded by a volume, typically a box,
defined by the 3D printing device.

\paragraph{Obtaining Models}
A 3D model is created as the result of a manual or an automatic 3D
modeling process. A manual modeling process is similar to plastic
arts such as sculpting. Nevertheless, specialized software tools are
always invloved in the final creation of 3D models. A 3D model can
be algorithmically computed via an automatic procedural process, such
as constructive solid modeling
Constructive solid geometry allows a modeler to create a complex surface or object by using Boolean operators to combine objects.

or it
can be generated from scanned data.

data exchange
% =============================================================================
% \bibliographystyle{abbrv}
% \bibliography{abrev,thickening}
% =============================================================================
\end{document}

A valid model
%% represented as a polygonal mesh comprises a polygon-soup (an arbitrary
%% set of polygons) that represents an artifact-free and sufficiently thick closed 2D-manifold, or a collection of closed 2D-manifolds that do not intersect each other. A valid printable model satisfies additional requirements imposed by various printing devices (different printing devices impose different requirements, as they are based on different technologies).

%% Correct and consistent representations of 3D objects are required by conventional applications, such as modeling, simulation, visualization, CAD (Computer-Aided Design), CAM (Computer-Aided Manufacturing), finite element analysis, and the like. However, the acquired 3D models, whether created by hand or by automatic tools, usually contain errors and inconsistencies. For example, they can contain wrongly-oriented, intersecting, or overlapping polygons, cracks, and T-junctions. In addition, polygons might be missing, and topological information can be inconsistent. Problems are caused by designer errors or software errors in the modeling tool. These errors can be compounded by data exchange problems, such as: (i) automatic transfer between CAD formats (e.g., IGES (Initial Graphics Exchange Specification), STEP (Standard for Exchange of Product model data), DXF (Drawing Interchange Format), binary files from CATIA8 (Computer Aided Three-dimensional Interactive Application) or AutoCAD), between B-Spline or NURBS-based (Non-uniform Rational B-Spline) formats; (ii) geometric transformation into an engineering analysis system (e.g., a triangular surface mesh). The techniques to reconstruct manifold models from acquired 3D models vary according to efficiency, robustness, level of automation, and other preconditions. Level of automation is measured in terms of required user input. Preconditions include requirements, such as: (i) polygons in the input set are consistently oriented; (ii) most polygons are orthogonal; or (iii) the input parts are closed 2D-manifolds. A na�ve but efficient approach is to use scene-relative tolerances to "fill over'' cracks in the model or merge features within some tolerance. Boundary-based approaches that try to infer solid structures from how input polygons mesh together are likely to perform incorrectly in the presence of non-manifold geometry. All approaches do not work well when the size of errors are larger than the smallest feature in the model. As degenerate input is commonplace in practical applications and numerical errors are inevitable, an algorithm implemented while ignoring limited precision imposed by floating-point arithmetic get into an infinite loop, or just crash, while running on a degenerate, or nearly degenerate, input; see [KMP+08,Sch00] for examples.





%% The reconstruction of precise surfaces from unorganized point clouds derived from laser scanner data or photogrammetric image measurements is a very hard problem, not completely solved and problematic in case of incomplete, noisy and sparse data. The generation of polygonal models that can satisfy high modeling and visualization demands is required in different applications, like video-games, movies, virtual reality applications, e-commerce and other graphics applications. The goal is always to find a way to create a computer model of an object which best fit the real object. Polygons are usually the ideal way to accurately represent the results of measurements, providing an optimal surface description. While the generation of digital terrain models has a long tradition and has found efficient solutions, the correct modeling of closed surfaces or free-form objects is of recent nature, a not completely solved problem and still an important issue investigated in many research activities.


%% Surface reconstruction from images is an old, fundamental yet difficult problem in computer vision, which has been extensively investigated over the past three decades. Recent developments in camera calibration and multimedia computing have broadened the interests in the reconstruction problem, and the big proliferation of digital cameras and computers allow to take and process multiple images respectively. The emphasis of the nowaday applications has been shifted to generating more "appearance" views of the scene, and it requires highly detailed surfaces to be constructed within a tolerable computational resources. Such a task, however, is difficult to accomplish due to the intrinsic ill-posedness of the reconstruction, interference from image noise, and insufficiency of scalability and flexibility.


%% Increasing need for geometric 3D models
%% �
%% Movie industry, games, virtual environments...
%% � Existing solutions are not fully satisfying
%% �
%% User-driven modeling: long and error-prone
%% �
%% 3D scanners: costly and cumbersome
%% � Alternative: analyzing image sequences
%% �
%% Cameras are cheap and lightweight
%% �
%% Cameras are precise (several megapixels)


%% retrieve the 3D shape

%% We proposed a new method for reconstruct 3D surface from (scanned) multiple-view 2D images. 


%% A point cloud is a set of data points in some coordinate system.

%% In a three-dimensional coordinate system, these points are usually defined by X, Y, and Z coordinates, and often are intended to represent the external surface of an object.

%% Point clouds may be created by 3D scanners. These devices measure in an automatic way a large number of points on the surface of an object, and often output a point cloud as a data file. The point cloud represents the set of points that the device has measured.

%% While point clouds can be directly rendered and inspected,[1] usually point clouds themselves are generally not directly usable in most 3D applications, and therefore are usually converted to polygon mesh or triangle mesh models, NURBS surface models, or CAD models through a process commonly referred to as surface reconstruction.




%% Opposite:
%%  modeled the prototypical shape of an object, seeking models that were invariant to
%% changes in color, texture, and minor within-class shape deformation.
